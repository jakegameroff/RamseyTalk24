%! TeX root: ../main.tex
\section{Introduction}
Ramsey Theory explores the underlying structure emerging in ``large enough" complex systems. For example, in 1930 Frank Ramsey proved that for each \( k \in \mathbb{N}  \) there is a sufficiently large \( n \in \mathbb{N}  \) such that in any red-blue coloring of the edges of the complete graph \( K_{n} \) there is a set of \( k \) vertices joined by edges of the same color \cite{Ramsey30}.

Another seminal result is Van der Waerden's theorem \cite{Waerden27}, which states that for all positive integers \( r, k \in \mathbb{N}  \), there is a large enough \( n \in \mathbb{N}  \) such that if we color the integers in \( [n] \coloneqq \{ 1,2,\hdots ,n \}  \) using \( k \) colors, one can always find a set of \( r \) monochromatic integers in arithmetic progression (i.e. there is a number \( d \geq 0 \) such that the difference of all consecutive terms is \( d \)).

Therefore, Ramsey Theory essentially implies that \emph{complete disorder is impossible} in large enough systems.
