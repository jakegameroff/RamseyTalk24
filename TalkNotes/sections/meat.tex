%! TeX root: ../main.tex
\section{Ramsey's Theorem}
We have already seen the finite version of Ramsey's theorem in the introduction. Here is its statement:
\begin{theorem}
For each \( k \in \mathbb{N} \) there is a positive integer \( n = R(k) \) such that in any red-blue coloring of the edges of \( K_{n}  \) there is a set \( X \) of \( k \) vertices such that all edges between vertices in \( X \) are the same color.
\end{theorem}
Pondering a little bit, a natural question may arise: \emph{is there a similar kind of structure when coloring the edges of an infinite graph?} It turns out that the answer is yes, and this result is an infinite version of Ramsey's theorem.

For a set \( X \), let \( [X]^{2} = \{ A \subseteq X : |A| = 2 \}  \) be the collection of two element subsets of \( X \). We are now ready to state and prove the (infinite) Ramsey theorem.
\begin{theorem}[Ramsey's Theorem \cite{Misra12}]
	Let \( k \in \mathbb{N}  \). Then for every \( k \)-coloring of \( [\mathbb{N} ]^{2}  \) there is an infinite subset \( A \subseteq N		 \) of naturals such that every pair in \( [A]^{2}  \) is the same color.
\end{theorem}
\begin{proof}
Define \( A_0 = \mathbb{N}  \) and fix a point \( x_0 \in A_0 \). Since \( x_0 \) is incident to infinitely many edges which are each either red or blue, it follows from the Pigeonhole principle that there is a color \( c_0 \in \{ \mbox{red, blue} \}  \) such that \( x_0 \) is incident to infinitely many edges with color \( c_0 \).

We define the set \( A_1 = \{ x \in A_0 : \{ x_0, x \} \mbox{ has color \( c_0 \)} \}  \). Now fix \( k \geq 2 \) and assume the sets \( A_1, A_2, \hdots ,A_{k-1}  \) have been defined and are infinite. Fix a point \( x_{k-1}  \in A_{k-1}  \). Note that \( A_{k-1} \) is infinite, so by this same Pigeonhole argument we can find a color \( c_{k-1}  \in \{ \mbox{red, blue} \}  \) so that \( x_{k-1}  \) has infinitely many neighbours \(y \in A_{k-1}  \) so that \( \{ x_{k-1} ,y \}  \) has color \( c_{k-1}  \). Correspondingly, we define the set \( A_{k} = \{ x \in A_{k-1}  : \{ x_{k-1} , x \} \mbox{ has color \( c_{k-1} \)} \}  \), which is infinite by construction.

Continuing this way, we obtain sequences \( (A_{k}) \), \( (x_{k})  \), and \( (c_{k}) \) with the following three properties for every \( k \in \mathbb{N}  \): 
\begin{enumerate}[leftmargin=1.2cm]
	\item \( x_{k} \in A_{k}  \)
	\item \( A_{k}  \) is infinite
	\item \( A_{k+1} \subseteq A_{k}  \) 
\end{enumerate}
Now let \( c \in \{ \mbox{red, blue} \}  \) be a color such that \( c = c_{k}  \) for infinitely many \( k \) (applying the Pigeonhole argument one last time). Hence the set \[ A = \{ x_{k} : c_{k} = c \}  \] is infinite. We claim that all edges between points in \( A \) are of color \( c \). Indeed, if \( \{ x_{i} , x_{j}  \} \in [A]^{2}  \) with \( i < j \) then from property (3) we have \( A_{j} \subseteq A_{j-1} \subseteq \cdots \subseteq A_{i}  \) so that \( x_{j} \in A_{i}  \). By the construction of \( A_{i}  \), the edge \( \{ x_{i} , x_{j}  \}  \) has color \( c_{i} = c \). Since \( x_{i}  \) and \( x_{j}  \) were arbitrary, the proof is complete.
\end{proof}
\section{Applications to Analysis}
Although this theorem is interesting in its own right, it provides us with a really elegant way to prove the famous Bolzano-Weierstrass theorem. I must note that I learnt about this technique in Professor Anush Tserunyan's analysis class Math 254 during my first semester here at McGill.

We will first use the following lemma:
\begin{lemma}
Every sequence of reals has a monotone subsequence.
\end{lemma}
\begin{proof}
	Given a sequence \( (x_{n}) \subseteq \mathbb{R} \) and natural numbers \( n , m  \) with \( n < m \) , we color the edge \( e = \{n , m \}  \) red if \( x_{n} \leq x_{m}  \) and we color \( e \) blue otherwise. Then (theorem) implies that there is an infinite subset \( A \subseteq \mathbb{N}  \) such that \( [A]^{2}  \) is either red or blue. Write \( A = (n_{k} )_{k \in \mathbb{N} }  \). Then the subsequence \( (x_{n_{k} })_{k \in \mathbb{N}}  \) is monotone: it is non-decreasing if \( [A]^{2}  \) is red and it is non-increasing if \( [A]^{2}  \) is blue.
\end{proof}
We now state and prove the Bolzano-Weierstrass theorem.
\begin{theorem}[Bolzano-Weierstrass]
Every bounded sequence of real numbers has a convergent subsequence.
\end{theorem}
\begin{proof}
Let \( (x_{n}) \) be a bounded sequence of real numbers. Then from (lemma), it has a monotone subsequence \( (x_{n_{k} })_{k \in \mathbb{N} } \). Since \( (x_{n}) \) is bounded, so is \( (x_{n_{k} })_{k \in \mathbb{N}}  \). Since bounded and monotone, it must be convergent as per the monotone convergence theorem.
\end{proof}
